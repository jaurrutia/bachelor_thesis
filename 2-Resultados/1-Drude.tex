\chapter{Resultados numéricos}

Para estudiar la respuesta electromagnética (EM) de una monocapa de nanopartículas (NPs) inmersa en un medio dieléctrico, denominado matriz, y soportada sobre un sustrato dieléctrico, se emplea el formalismo del modelo de esparcimiento coherente (Coherent Scattering Model, CSM) para calcular la reflectancia $R$ y transmistancia $T$ del sistema \cite{reyes2018analytical}. El CSM proporciona expresiones analíticas de los coeficientes de amplitud de reflexión $r$ y transmisión $t$ de la monocapa cuando está suspendida en el espacio libre (Free Stanting Monolayer, FSM) [Ecs. \eqref{eqs:rtcoh}], y para el sistema matriz-monocapa-sustrato tanto en incidencia externa [Ecs. \eqref{eqs:rtCSMext}] como interna, o bien en una configuración de reflexión total atenuada \index{Reflexión total!atenuada} (Attenuated Total Reflection, ATR) [Ecs. \eqref{eqs:rtCSMATR}]. En la primera sección de este capítulo se calcula la respuesta EM de una moncapa de NPs esféricas considerando que la función dieléctrica de las NPs que conforman la monocapa está dada por el modelo de Drude-Sommerfeld [Ec. \eqref{eq:Drude}], que depende de sólo dos parámetros: la frecuencia de plasma $\omega_p$, que sintoniza las resonancias plasmónicas de superficie (Surface Plasmon Resonances, SPRs), y la constante fenomenológica de amortiguamiento $\gamma$, que ajusta el ancho de cada SPR. En caso de identificar en el cálculo de la refectancia y transmitancia de una monocapa de NPs excitaciones distintas a las  SPRs de partículas individuales (Single Particles SPR, SP-SPRs), como el \emph{modo guiado} reportado en \cite{kabashin2009plasmonic} y \cite{danilov2018ultra}, también denominado resonancia de red de plasmón de superfice (Surface Plasmon Lattice Resonance, SPLR), la elección de $\omega_p$ y $\gamma$ evita el traslape entre ambas excitaciones y facilita la identificación de cada modo. En la segunda sección se emplean las correcciones por tamaño para partículas esféricas de las funciones dieléctricas del oro y de la plata, para identificar si un modo semejante al modo guiado se encuentra en monocapas formadas con NPs de materiales reales, así como las  características de la monocapa para que pueda ser empleada en el biosensado.

\section{Resultados con el modelo de Drude-Sommerfeld}
\label{section:Drude}


 En la primera subsección se analiza la reflectancia de una FSM empleando el modelo de Drude-Sommerfled con parámetros $\hbar\omega_p = 4.3$ eV y  $\hbar\gamma = 0.15$ eV [ver Fig. \ref{sfig:Drude4eV}], y comparando la respuesta EM de la monocapa con la de una partícula individual. En la segunda subsección se estudia la reflectancia de una monocapa soportada en configuración de reflexión interna atenuada, ver Fig. \ref{fig:ATR}, empleando el modelo de Drude-Sommerfeld en un primer caso con los parámetros  $\hbar\omega_p = 4.3$ eV y  $\hbar\gamma = 0.15$ eV, y posteriormente con $\hbar\omega_p = 10$ eV y $\hbar\gamma = 0.15$ eV [ver Fig. \ref{sfig:Drude10eV}] para modificar la longitud de onda de las SP-SPRs; ulteriormente se calcula la reflectancia de la monocapa considerando  variaciones en la fracción de cubierta $\Theta$ y el radio $a$ de las NPs, parámetros que modifican las proporciones del sistema, tales como la distancia promedio entre las NPs, y la cantidad de electrones libres sobre la monocapa. Adicional al cálculo de la reflectancia, se calculó la transmitancia de la monocapa para las dos funciones dieléctricas, con $\hbar\omega_p =4.3$ eV y $\hbar\omega_p =10$ eV, para corroborar que los modos distintos a las SP-SPRs tienen un comportamiento semejante a un modo guiado. 
	
	\subsection{Reflectancia de una monocapa suspendida en aire}
	\label{ssection:DrudeFSM}
	
	
Para el cálculo de la reflectancia mediante el CSM de una FSM suspendida en aire ($n_m=1$), se empleó la Ec.  \eqref{eq:R} con el coeficiente de amplitud de reflexión coherente $r_{coh}$ [Ec.  \eqref{seq:rcoh}].  En la Fig.  \ref{fig:R-FSM} se muestran los resultados de la reflectancia $R$ en función del ángulo de incidencia $\theta_i$ y tanto de la longitud de onda $\lambda$ del haz incidente (escala inferior), como de la energía del haz incidente en unidades de $\hbar\omega = h c /\lambda$ (escala superior).  La frecuencia de plasma empleada para la función dieléctrica tipo Drude fue $\hbar\omega_p = 4. 3$ eV y la constante fenomenológica de amortiguamiento $\hbar\gamma = 0. 15$ eV (que corresponden a $288. 5$ nm  y $8,270$ nm, respectivamente). Se consideraron NPs de radio $a=30$ nm y fracciones de cubierta $\Theta$: $0. 05$, $0. 1$, $0. 2$, $0. 3$ y $0. 4$. En el renglón superior de la Fig. \ref{fig:R-FSM}, gráficas de $\mathbf{i)}$ a $\mathbf{v)}$, se muestra la reflectancia para polarización \emph{p}, mientras que en el renglón inferior se presentan las gráficas para polarización \emph{s}, $\mathbf{vi)}$ -- $\mathbf{x)}$. La línea punteada vertical verde  en $\lambda \approx 526$ nm corresponde a la SP-SPR dipolar ($\ell = 1$), mientras que la línea vertical rosa punteada en $\lambda \approx 462$ nm corresponde a la excitación del modo cuadrupolar ($\ell=2$).
					
	\begin{figure}[h!]\centering
\includegraphics[width = .94\linewidth	]{2-Resultados/figs/4-Wp4FSMThetaVar/0-2D_Grid.png}%
\includegraphics[scale=.89, trim={00 -5 00 00}, clip]{2-Resultados/figs/0-RBar_v}\vspace*{-.5em}
	\caption{Gráficas de la reflectancia para una FSM como función del ángulo de incidencia $\theta_i$ y tanto de la longitud de onda $\lambda$ (escala inferior), como de la energía del haz incidente en unidades de $\hbar\omega$ (escala superior), para una función dieléctrica tipo Drude con $\hbar\omega_p=4. 3$ eV  y  $\hbar\gamma=0. 15$ eV.  Las gráficas   en el renglón superior [$\mathbf{i)-v)}$]  muestran los resultados de reflectancia para  polarización \emph{p} y las del renglón inferior  [$\mathbf{i)-v)}$] para polarización  \emph{s}, donde se consideraron NPs de radio $a=30$ nm y distintas fracciones de cubierta $\Theta$: $0. 05$, $0. 1$, $0. 2$, $0. 3$ y $0. 4$. Las líneas verticales punteadas verdes y rosas corresponden a las SP-SPRs dipolar ($526$ nm) y cuadrupolar ($462$ nm), respectivamente.}	\label{fig:R-FSM}	
	\end{figure}		
					
La reflectancia para polarización \emph{p} [Fig. \ref{fig:R-FSM} $\mathbf{i)-v)}$] es cero para el ángulo de Brewster $\theta_B \approx 45^\circ$ y para regiones alejadas de las SP-SPRs (líneas punteadas verticales verde y rosa  localizadas en $526$ nm y $462$ nm, respectivamente). En la gráfica \textbf{v)}, $\Theta=0.4$,  se observa a $526$ nm (escala inferior) y $\theta_i>50^\circ$ que $R\approx 0$ sin embargo, la reflectancia aumenta para longitudes de onda mayores y menores a $526$ nm. Conforme la fracción de cubierta disminuye, gráficas \textbf{iii)} y \textbf{iv)}, la extinción de luz a $526$ nm  es menos evidente y para las fracciones de cubierta $\Theta=0.1$ y $0.05$, gráficas \textbf{ii)} y \textbf{i)}, ya no es apreciable la extinción de luz a la frecuencia de la SP-SPR dipolar. En contraparte, para polarización \emph{s} [Fig. \ref{fig:R-FSM} $\mathbf{vi)-x)}$] la reflectancia es apreciable para todo ángulo de incidencia a las frecuencias de las SP-SPRs. 

Para comparar la respuesta EM  de una FSM al variar la fracción de cubierta, se grafican en la  Fig. \ref{fig:FSM-Cuts} cortes de la reflectancia para $\theta_i = 65^\circ$, ángulo en donde se extingue la luz reflejada al rededor de la SP-SPR dipolar para fracciones de cubierta $\Theta>0.2$ en la Fig. \ref{fig:R-FSM}. Se muestran cortes de la reflectancia tanto para un haz incidente con polarización \emph{p} $R_p$ [Fig. \ref{sfig:FSM-cutp}], como uno con polarización \emph{s} $R_s$ [Fig. \ref{sfig:FSM-cuts}]. Para la polarización \emph{p} se presenta un mínimo en la reflectancia alrededor de $526$ nm para fracciones de cubierta mayores a $\Theta = 0.05$. Los mínimos de $R_p$ a la frecuencia de la SP-SPR dipolar son más pronunciados conforme aumenta la fracción de cubierta sin embargo, para $\Theta=0.05$ se observa un máximo en lugar de un mínimo. Para polarización \emph{s}, se presenta un máximo en la reflectancia a $526$ nm para todos los valores de $\Theta$. Para las fracciones de cubierta mayores, $\Theta = 0.3$ y $\Theta = 0.4$,  se observa un  mínimo en la reflectancia alrededor de $462$ nm para ambas polarizaciones, lo que corresponde a la SP-SPR cuadrupolar.

		\begin{figure}[h!]\centering\hspace*{-1.5em}
	\begin{subfigure}{.01\linewidth}\caption{}\label{sfig:FSM-cutp}\vspace{4.5cm}\end{subfigure}
	\begin{subfigure}{.45\linewidth}\hspace*{-1.5em}
	\includegraphics[scale=1]{2-Resultados/figs/4-Wp4FSMThetaVar/cut_angle_65_p.pdf}\end{subfigure}
	\begin{subfigure}{.01\linewidth}\caption{}\label{sfig:FSM-cuts}\vspace{4.5cm}\end{subfigure}\hspace*{-1.em}
	\begin{subfigure}{.45\linewidth}\centering
	\includegraphics[scale=1 ]{2-Resultados/figs/4-Wp4FSMThetaVar/cut_angle_65_s.pdf}\end{subfigure}\vspace*{-.5em}
	\caption{Cortes de la Fig. \ref{fig:R-FSM} a $\theta_i = 65^\circ$ de reflectancia de una FSM de NPs esféricas de radio $a=30$ nm en polarización \textbf{a)} \emph{p} y \textbf{b)} \emph{s} como función tanto de la longitud de onda $\lambda$ (escala inferior), como de la energía del haz incidente en unidades de $\hbar\omega$ (escala superior). Los parámetros de la función dieléctrica tipo Drude para las NPs son $\hbar\omega_p = 4.3$ eV y $\hbar\gamma = 0.15$ eV y las fracciones de cubierta consideradas fueron $\Theta$: $0. 05$, $0. 1$, $0. 2$, $0. 3$ y $0. 4$. Las líneas verticales punteadas verdes y rosas corresponden a las SP-SPRs dipolar ($526$ nm) y cuadrupolar ($462$ nm), respectivamente.}\label{fig:FSM-Cuts}
	\end{figure}	

%Al calcular la distancia promedio $\langle d \rangle$ entre las NPs  de $a = 30$ nm mediante la Tab. \eqref{tab:MeanD}, se obtiene que $\langle d \rangle = 177.8$ nm para $\Theta = 0.05$, $\langle d \rangle = 105.1$ nm para $\Theta = 0.1$ y $\langle d \rangle = 24.1$ nm para $\Theta = 0.4$. El análisis de una partícula individual es válido para el caso de $\Theta=0.05$ (en negro en la Fig. \ref{fig:FSM-Cuts}), debido a la distancia promedio entre las NPs, por tanto, la presencia del máximo en la reflectancia en la SP-SPRs dipolar (línea punteada vertical verde) corresponde a una cota mínima en $R_p$ debido al esparcimiento de cada una de las NPs; en la Fig. \ref{sig:ScatDiag-Drude4} se grafica el diagrama de esparcimiento de una partícula esférica en la monocapa . Conforme el número de NPs aumenta, la luz esparcida también lo hace sin embargo, la extinción de luz es causada en su mayoría por la absorción, como se observa en la Fig. \ref{sfig:Q-ext-Drude4}, por lo que también se llega a una

En las Figs. \ref{fig:R-FSM} y \ref{fig:FSM-Cuts} se observa la respuesta EM de una monocapa de NPs suspendida en vacío al interactuar con una onda plana. Si se considera la presencia de un sustrato que soporte la monocapa, se puede estar en configuraci\'on de incidencia externa o interna, seg\'un sea el medio de incidencia de la onda plana. Para incidencia externa, a todo ángulo de incidencia,  una onda plana ilumina a las NPs de la monocapa por lo que, respecto al caso de la FSM, la posición de los máximos y mínimos de la reflectancia no cambiarán y los valores de $R$ presentarán un decrecimiento, debido al sustrato que disminuye el contraste entre índice de refracción de las NPs en la monocapa y el medio de transmisión. Por otro lado, para el caso de incidencia interna y ángulos mayores al ángulo crítico $\theta_c = \arcsin(n_m/n_s)$, las NPs en la monocapa son iluminadas por ondas evanescentes, por tratarse de un configuración en ATR, por lo que es posible  observar cambios en la respuesta EM de la monocapa, como sucede  cuando se tiene una placa continua y se excitan plasmones polaritones de superficie.

	\subsection{Reflectancia y transmitancia de una monocapa soportada sobre un sustrato en configuración de reflexión total atenuada}
	\label{ssection:DrudeATR}

La respuesta EM de una monocapa de NPs, suspendida en una matriz con índice de refracción $n_m$ y soportada por un sustrato con índice de refracción $n_s$, se calcula al emplear la Ec.  \eqref{eq:R} con el coeficiente de amplitud de reflexión $r$ de la Ec.  \eqref{seq:rCSMATR}. Para comparar los resultados de la reflectancia $R$ para una FSM y una monocapa en configuración ATR, se emplean los parámetros utilizado en los cálculos de las Figs. \ref{fig:R-FSM} y \ref{fig:FSM-Cuts} ($n_m=1$ , $a=30$ nm, $\hbar\omega_p=4.3$ eV y  $\hbar\gamma = 0.15$ eV) considerando un sustrato con índice de refracción $n_s=1.5$, pues el índice de refracción del vidrio BK7 es $n_{BK7}=1.50\pm 0.05$ en un intervalo de longitud de onda entre $334.1$ nm y $2,325.4$ nm \cite{schott2019datasheet}. En la Fig.  \ref{fig:R-ATR4} se presentan los resultados de la reflectancia $R$ en función del ángulo de incidencia $\theta_i$ y tanto de la longitud de onda $\lambda$ del haz incidente (escala inferior) como de la energía del haz $\hbar\omega$ (escala superior). Las gráficas \textbf{i) -- v}) en la Fig. \ref{fig:R-ATR4}  corresponden a la polarización \emph{p}, mientras que las gráficas \textbf{vi) -- x)} a polarización \emph{s}. Al igual que para la FSM, se consideraron los casos para la fracción de cubierta $\Theta = 0.05,\,0.1,\,0.2,\,0.3$ y $0.4$. Las SP-SPRs corresponden a la línea vertical verde punteada en $\lambda \approx 526$ nm para el modo dipolar y la línea vertical rosa punteada en  $\lambda \approx 462$ nm para el modo cuadrupolar. Adicionalmente, los puntos amarillos en la Fig. \ref{fig:R-ATR4} corresponden a los mínimos en $R$ para ángulos mayores al ángulo crítico entre el sustrato y la matriz ($\theta_c\approx 41.8^\circ$) y longitudes de onda mayores a la SP-SPR dipolar.

	\begin{figure}[h!]\centering
\includegraphics[width = .94\linewidth, trim={00 00 00 00}, clip	]{2-Resultados/figs/1-Wp4ThetaVar/0-2D_Grid}%
\includegraphics[scale=.89, trim={00 -5 00 00}, clip]{2-Resultados/figs/0-RBar_v}\vspace*{-.5em}
	\caption{Gráficas de la reflectancia de una monocapa en configuración ATR como función del ángulo de incidencia $\theta_i$ y de la longitud de onda $\lambda$ (escala inferior) así como de la energía del haz incidente en unidades de $\hbar\omega$ (escala superior), para una función dieléctrica tipo Drude con $\hbar\omega_p=4. 3$ eV  y  $\hbar\gamma=0. 15$ eV.  Las gráficas   en el renglón superior [$\mathbf{i)-v)}$]  muestran los resultados de reflectancia para  polarización \emph{p} y las del renglón inferior  [$\mathbf{vi)-x)}$] para polarización  \emph{s}, donde se consideraron NPs de radio $a=30$ nm y distintas fracciones de cubierta $\Theta$: $0. 05$, $0. 1$, $0. 2$, $0. 3$ y $0. 4$. Las líneas verticales punteadas verdes y rosas corresponden a las SP-SPRs dipolar ($526$ nm) y cuadrupolar ($462$ nm), respectivamente.	os puntos amarillos corresponden a los mínimos en $R$ para ángulos mayores a $\theta_c\approx 41.8^\circ$ y longitudes de onda mayores a la SP-SPRs dipolar.}	\label{fig:R-ATR4}	
	\end{figure}	

En la Fig.  \ref{fig:R-ATR4} se observa que $R\approx 1$ para ángulos mayores al ángulo crítico, $\theta_c \approx 41.8^\circ $ excepto en dos regiones: a las longitudes de onda correspondientes a las SP-SPRs (líneas punteadas verticales) y en una región a longitudes de onda mayores a la SP-SPR dipolar (puntos amarillos). La disminución en la reflectancia después del ángulo crítico alrededor de las SP-SPRs es resultado de la extinción de luz debido a la presencia de las NPs y, al considerar la interacción entre ellas, así como con el sustrato, puede presentarse un corrimiento al rojo o al azul de la SPR que depende del ángulo de incidencia del haz, como se observa en las gráficas \textbf{iii)--v)} en donde las SP-SPRs se corren al rojo para $\theta_i\approx \theta_c$ y al azul para $\theta_i\approx 90^\circ$. La extinción a las longitudes de onda cercanas a las SP-SPRs es más evidente para las fracciones de cubierta más grandes consideradas. Por ejemplo, en el panel superior de la Fig. \ref{fig:R-ATR4},  la excitación cuadrupolar de una sola partícula es apreciable en $\lambda \approx 4,620$ nm cuando $R_p$ disminuye para $\Theta = 0.4$, gráfica  $\mathbf{v)}$, en comparación a $\Theta = 0.05$, $\mathbf{i)}$. A pesar de que este comportamiento es análogo para la polarización \emph{s}, panel inferior de la  Fig. \ref{fig:R-ATR4}, los valores de $R_s$ a las frecuencias de las SP-SPRs son mayores que los de $R_p$, como se observa al comparar las gráficas de los cálculos con $\Theta=0.3$: \textbf{iv)} para $R_p$ y \textbf{ix)} para $R_s$.

Adicional a la región cercana a las SP-SPRs, se observan mínimos en la reflectancia para ángulos de incidencia mayores al ángulo crítico y para longitudes de onda mayores a la SP-SPR dipolar, los cuales están  representados por los puntos amarillos en la Fig. \ref{fig:R-ATR4}. Dado que los puntos amarillos corresponden a una excitación que ocurre energías  menores en comparación a las SP-SPRs, ésta no puede ser plasmónica de partícula individual,  por lo que especula que se debe a una respuesta colectiva como la PSLR reportada en \cite{danilov2018ultra}. Al comparar las gráficas en la  Fig.  \ref{fig:R-ATR4} se observa que la posible excitación colectiva se corre al rojo  conforme aumenta la fracción de cubierta $\Theta$; comportamiento más evidente en polarización \emph{p} que en \emph{s} [ver \textbf{v)} y \textbf{x)}]. Para continuar con el análisis de la presunta excitación colectiva se compara la reflectancia a $\theta_i = 65^\circ$, mismo ángulo de inidencia empleado en la Fig. \ref{fig:FSM-Cuts}.
    
  En la Fig. \ref{fig:R-ATR4-Cuts} se presentan cortes a $\theta_i = 65^\circ$ de la reflectancia graficada en la Fig. \ref{fig:R-ATR4} para todas las fracciones de cubierta consideradas; las líneas punteadas verticales corresponden a las longitudes de onda de las SP-SPRs (verde para la excitación dipolar y rosa para la cuadrupolar). En polarización \emph{p}, Fig. \ref{sfig:R-ATR4-cutp}, la excitación de la monocapa para $\Theta=0.05$ alrededor de $\lambda \approx 462$ nm coincide con la SP-SPR cuadrupolar y conforme la fracción de cubierta aumenta, la excitación de la monocapa presenta un corrimiento al azul. En polarización \emph{s}, Fig. \ref{sfig:R-ATR4-cuts},  la SP-SPR cuadrupolar se observa en la respuesta de la monocapa para todas las fracciones de cubierta. La reflectancia, para ambas polarizaciones, en la longitud de onda de la SP-SPR cuadrupolar disminuye conforme la fracción de cubierta crece, por lo que se relaciona con la cantidad de NPs presentes en la monocapa. A diferencia de la SP-SPR cuadrupolar, la excitación dipolar de partícula individual ($526$ nm) no se aprecia para todos los casos estudiados en la Fig. \ref{fig:R-ATR4-Cuts}. En la respuesta óptica de la monocapa sólo se presenta una excitación cercana a la SP-SPR dipolar para los resultados de $R_p$ y para las fracciones de cubierta $\Theta = 0.1,\,0.2,\,0.3$ y $0.4$, la cual se corre al azul conforme aumenta la fracción de cubierta, siendo el mayor corrimiento de $25$ nm al azul para $\Theta = 0.4$. Al igual que la SP-SPRs cadrupolar, la excitación de dipolar se corre al azul conforme $\Theta$ aumenta. Sin emabargo, se observa una excitación a longitudes de onda mayores a la SP-SPR dipolar para ambas polarizaciones.   

\begin{figure}[h!]\centering\hspace*{-1.5em}
	\begin{subfigure}{.01\linewidth}\caption{}\label{sfig:R-ATR4-cutp}\vspace{4.5cm}\end{subfigure}
	\begin{subfigure}{.45\linewidth}\hspace*{-1.5em}
	\includegraphics[scale=1]{2-Resultados/figs/1-Wp4ThetaVar/cut_angle_65_p.pdf}\end{subfigure}
	\begin{subfigure}{.01\linewidth}\caption{}\label{sfig:R-ATR4-cuts}\vspace{4.5cm}\end{subfigure}\hspace*{-1.em}
	\begin{subfigure}{.45\linewidth}\centering
	\includegraphics[scale=1 ]{2-Resultados/figs/1-Wp4ThetaVar/cut_angle_65_s.pdf}\end{subfigure}\vspace*{-.5em}
	\caption{Cortes de la Fig. \ref{fig:R-ATR4} a $\theta_i = 65^\circ$ de reflectancia de una monocapa en configuración ATR de NPs esféricas de radio $a=30$ nm en polarización \textbf{a)} \emph{p} y \textbf{b)} \emph{s} como función de la longitud de onda $\lambda$ (escala inferior) y de la energía $\hbar\omega$ (escala superior). Los parámetros de la función dieléctrica tipo Drude para las NPs son $\hbar\omega_p = 4.3$ eV y $\hbar\gamma = 0.15$ eV y las fracciones de cubierta consideradas fueron $\Theta$: $0. 05$, $0. 1$, $0. 2$, $0. 3$ y $0. 4$. Las líneas verticales punteadas verdes y rosas corresponden a las SP-SPRs dipolar ($526$ nm) y cuadrupolar ($462$ nm), respectivamente. }\label{fig:R-ATR4-Cuts}
	\end{figure}	  

Los mínimos a $\lambda>530$ nm, que corresponden al presunto modo colectivo, presentan un corrimiento al rojo conforme la fracción de cubierta de la monocapa aumenta  para ambas polarizaciones, contrario al comportamiento observado en las excitaciones de la monocapa cercanas a las SP-SPRs. Otra diferencia entre las excitaciones en $\lambda$ mayores a las SP-SPRs y los corrimientos al azul de éstas, es que la disminución en el valor de $R$ no es monotona, sino que el decrecimiento en $R$ es máximo a fracciones de cubierta medias. Por lo anterior, los mínimos en $R_p$ y $R_s$ localizados a longitudes de onda mayores a la de los modos plasmónicos de partícula individual no son corrimientos de las excitaciones multipolares de una partícula, sino una probable respuesta colectiva de las NPs en la monocapa, que se separa de las SP-SPRs en mayor medida para la polarización \emph{p} que para \emph{s}.

En la Fig. \ref{sfig:R-ATR4-cutp} se observa que para $\Theta = 0.4$, el presunto modo colectivo se separa de la SP-SPRs dipolar (línea punteada vertical verde) $320$ nm, por lo que se sintoniza en $\lambda=830$ nm, mientras que en la Fig. \ref{sfig:R-ATR4-cuts}, el presunto modo colectivo se sintoniza en $\lambda = 640$ nm, separándose de la SP-SPR dipolar $120$ nm. Si el presunto modo colectivo depende no sólo de la fracción de cubierta de la monocapa, sino también del material de las NPs, su posición cambiará al incrementar la frecuencia de plasma en el modelo de Drude, que caracteriza la respuesta EM de las NPs y modifica la posición de las SP-SPRs: al considerar $\hbar\omega_p = 10$ eV las SP-SPRs se corren al azul. Los resultados de la reflectancia de un sistema monocapa con los parámetros empleados en la Fig. \ref{fig:R-ATR4}, pero con $\hbar\omega_p = 10$ eV, se muestran en la Fig. \ref{fig:R-ATR10}. Adicional a la SP-SPR dipolar y cuadrupolar (líneas verticales punteadas verde y rosa en $265$ nm y $211$ nm, respectivamente), la SP-SPR octopolar ($\ell = 3$) se muestra en las gráficas mediante la línea vertical punteada cian en $195$ nm, al igual que el presunto modo colectivo que corresponde a los puntos amarillos.
	
	\begin{figure}[h!]\centering
\includegraphics[width = .94\linewidth]{2-Resultados/figs/2-Wp10ThetaVar/0-2D_Grid.png}%
\includegraphics[scale=.89, trim={00 -5 00 00}, clip]{2-Resultados/figs/0-RBar_v}\vspace*{-.5em}
	\caption{Gráficas de la reflectancia de una monocapa en configuración ATR como función del ángulo de incidencia $\theta_i$ y de la longitud de onda $\lambda$ (escala inferior), así como de la energía del haz incidente en unidades de $\hbar\omega$ (escala superior), para una función dieléctrica tipo Drude con $\hbar\omega_p=10$ eV  y  $\hbar\gamma=0. 15$ eV.  Las gráficas   en el renglón superior [$\mathbf{i)-v)}$] muestran los resultados  para  polarización \emph{p} y las del renglón inferior  [$\mathbf{vi)-x)}$] para polarización  \emph{s}, donde se consideraron NPs de radio $a=30$ nm y distintas fracciones de cubierta $\Theta$: $0. 05$, $0. 1$, $0. 2$, $0. 3$ y $0. 4$. Las líneas verticales punteadas verdes, rosas y cianes corresponden a las SP-SPRs dipolar ($265$ nm), cuadrupolar ($211$ nm) y octopolar ($195$ nm), respectivamente.  Los puntos amarillos corresponden a los mínimos en $R$ para ángulos mayores a $\theta_c\approx 41.8^\circ$ y longitudes de onda mayores a la SP-SPRs dipolar. }	\label{fig:R-ATR10}	
	\end{figure}				
		
En las gráficas mostradas en la Fig. \ref{fig:R-ATR10} ($\hbar\omega_p = 10$ eV) se aprecian características semejantes a las observadas en la Fig. \ref{fig:R-ATR4} donde se empleó $\hbar\omega_p = 4.3$ eV. En ambos casos la reflectancia disminuye para $\theta_i>\theta_c=41.8^\circ$ y valores de $\lambda$ cercanos a las SP-SPRs (líneas verticales punteadas), así como en longitudes de onda mayoes a la SP-SPR dipolar, es decir, en la presunta excitación colectiva  (puntos amarillos); de igual forma, el corrimeinto al rojo de la presunta excitación colectiva respecto a la SP-SPR dipolar es mayor para polarización \emph{p} que para \emph{s}. Sin embargo, sólo para $\hbar\omega_p = 10$ eV, para los casos de polarización \emph{p} y $\Theta = 0.2,\,0.3$ y $0.4$, gráficas \textbf{iii)} a \textbf{v)} en la Fig. \ref{fig:R-ATR10}, se observa una tercera región con mínimos locales en la reflectancia, localizada en valores de $\lambda$ menores a la SP-SPR cuadrupolar y a distintos ángulos de incidencia, la cual adopta la forma de una ramificación que se corre al azul conforme el ángulo de incidencia aumenta. Asimismo, al modificar el parámetro $\hbar\omega_p$ de $4.3$ eV a $10$ eV se sintonizó la presunta excitación colectiva a longitudes de onda menores, por ejemplo, para $\Theta = 0.4$ a $\theta_i\approx 75^\circ$ la presunta excitación colectiva se presenta al rededor de $800$ nm para $\hbar\omega_p=4.3$ eV y al rededor de $600$ nm para $\hbar\omega_p=10$ eV, como se puede observar en las gráficas \textbf{v)} de las Figs. \ref{fig:R-ATR4} y \ref{fig:R-ATR10}.

Dado que la elección del parámetro $\omega_p$ sintoniza tanto las SP-SPRs como el presunto modo colectivo, la separación entre estos puede modificarse. Para comparar con el caso de $\hbar\omega_p=4.3$ eV [Fig. \ref{fig:R-ATR4-Cuts}], se grafica en la Fig. \ref{fig:R-ATR10-Cuts} cortes de la reflectancia graficada en la Fig. \ref{fig:R-ATR10}, donde se emplea $\hbar\omega_p = 10$ eV, a $\theta_i = 65^\circ$ para ambas polarizaciones,  Fig. \ref{sfig:R-ATR10-cutp} para \emph{p} y Fig. \ref{sfig:R-ATR10-cuts} para \emph{s}, en  función de la longitud de onda, para una monocapa de NPs de radio $a= 30$ nm  y fracciones de cubierta consideradas en la Fig. \ref{fig:R-ATR10}; las líneas punteadas verde, rosa y cian corresponden a las SP-SPRs dipolar, cuadrupolar y octopolar, respectivamente. Para ambas polarizaciones y para todas las fracciones de cubierta, se presenta una excitación a la longitud de onda correspondiente a la SP-SPR octopolar, al igual que un corrimiento al azul de la SP-SPR cuadrupolar. Adicional a estas excitaciones, se o aprecia un mínimo en $R_p$ y un punto de inflexión en $R_s$ a $\lambda\approx 185$ nm para todos los valores de $\Theta$ considerados; esta longitud de onda corresponde a la SP-SPR de orden $\ell = 4$ para una partícula de las que conforman a la monocapa.

\begin{figure}[h!]\centering\hspace*{-1.5em}
	\begin{subfigure}{.01\linewidth}\caption{}\label{sfig:R-ATR10-cutp}\vspace{4.5cm}\end{subfigure}
	\begin{subfigure}{.45\linewidth}\hspace*{-1.5em}
	\includegraphics[scale=1]{2-Resultados/figs/2-Wp10ThetaVar/cut_angle_65_p_Stack.pdf}\end{subfigure}
	\begin{subfigure}{.01\linewidth}\caption{}\label{sfig:R-ATR10-cuts}\vspace{4.5cm}\end{subfigure}\hspace*{-1.em}
	\begin{subfigure}{.45\linewidth}\centering
	\includegraphics[scale=1 ]{2-Resultados/figs/2-Wp10ThetaVar/cut_angle_65_s.pdf}\end{subfigure}\vspace*{-.5em}
	\caption{Cortes de la Fig. \ref{fig:R-ATR10} a $\theta_i = 65^\circ$ de reflectancia de una monocapa en configuración ATR de NPs esféricas de radio $a$ en polarización \textbf{a)} \emph{p} y \textbf{b)} \emph{s} como función de la longitud de onda $\lambda$ (escala inferior) y de la energía en unidades de $\hbar\omega$ (escala superior). Los parámetros de la función dieléctrica tipo Drude para las NPs son $\hbar\omega_p = 10$ eV y $\hbar\gamma = 0.15$ eV y las fracciones de cubierta consideradas fueron $\Theta$: $0. 05$, $0. 1$, $0. 2$, $0. 3$ y $0. 4$. Las líneas verticales punteadas verdes, rosas y cianes corresponden a las SP-SPRs dipolar ($265$ nm), cuadrupolar ($211$ nm) y octopolar ($195$ nm), respectivamente.  }\label{fig:R-ATR10-Cuts}
	\end{figure}	

La resonancia dipolar de una partícula individual no se alcanza a distinguir en la Fig. \ref{fig:R-ATR10-Cuts}, $\theta_i=65^\circ$, para ninguna polarización sin embargo, los mínimos en la reflectancia en $\lambda > 265$ nm se atribuyen a una respuesta colectiva. Las excitaciones del presunto modo colectivo con $\hbar\omega_p = 10$ eV se comportan de manera análoga al caso de $\hbar\omega_p = 4.3$ eV: se corren al rojo conforme aumenta la fracción de cubierta y su presencia es más evidente para fracciones de cubierta media, siendo  $\Theta=0.2$ para polarización \emph{p} y $\Theta=0.3$ para polarización \emph{s} cuando la reflectancia en la excitación del presunto modo colectivo alcanza el valor mínimo de reflectancia. Cuando $\Theta = 0.05$ (líneas negras en la Fig. \ref{fig:R-ATR10-Cuts}) la excitación del presunto modo colectivo se separa de la SP-SPR dipolar   $20$ nm y $15$ nm para polarización \emph{p} y \emph{s}, respectivamente, por lo que la respuesta colectiva engloba a la excitación dipolar de una partícula. Para $\Theta = 0.4$ la excitación del presunto modo colectivo se separa de la SP-SPR dipolar (línea verticual punteada verde) $290$ nm y $100$ nm para polarización \emph{p} y \emph{s}, respectivamente, es decir, que la separación entre ambas excitaciones es menor que cuando se consideró $\hbar\omega_p = 4.3$ eV en la Fig. \ref{fig:R-ATR4-Cuts} mas la anchura a media altura (Full Width at Half Maximum, FWHM) $\Delta\lambda_{FWHM}$ de la excitación es mayor. Por ejemplo, para $\Theta = 0.04$ y $\hbar\omega_p = 4.3$ eV,  $\Delta\lambda_{FWHM}\approx 200$ nm para polarización \emph{p} y  $\Delta\lambda_{FWHM}\approx 100$ nm para  \emph{s} (ver Fig. \ref{fig:R-ATR4-Cuts}), mientras que para  $\hbar\omega_p = 10$ eV,  $\Delta\lambda_{FWHM}\approx 300$ nm para polarización \emph{p} y  $\Delta\lambda_{FWHM}\approx 130$ nm para  \emph{s} (ver Fig. \ref{fig:R-ATR10-Cuts}).

Para polarización \emph{p}, adicional a las SP-SPRs de orden $\ell = 2,\,3$ y $4$ se observa una excitación que se localiza en la longitud de onda de la SP-SPR octopolar para $\Theta=0.05$ y se corre al azul conforme aumenta la fracción de cubierta, hasta localizarse en $\lambda= 135$ nm para $\Theta=0.4$ [ver línea vertical punteada azul en laFig. \ref{sfig:R-ATR10-cutp}]. La excitación a $\lambda<185$ en polarización \emph{p} tomo un valor mínimo en la reflectancia cuando $\Theta=0.2$, una fracción de cubierta media. Esta excitación corresponde a la ramificación también presente en las gráficas \textbf{iii)} a \textbf{v)} de la Fig. \ref{fig:R-ATR10}. Ya que se observó una segunda respuesta que no corresponde a las SP-SPRs, sino que sigue las tendencias del presunto modo colectivo, se especula que es un complemento  de ellas, dado que las  PSLRs reportadas en \cite{danilov2018ultra} son, para un sistema determinado, dos excitaciones que se corren al rojo y al azul. Es decir, que tanto el modo que se corre al azul a $\lambda<185$ nm [línea punteada azul en la Fig. \ref{sfig:R-ATR10-cutp}], como el presunto modo colectivo (puntos amarilos en la Fig. \ref{fig:R-ATR10}) se pueden relacionar con las PSLRs.  

 Ya que el presunto modo colectivo sufre un corrimiento al rojo al aumentar la fracción de cubierta, se analizó si el comportamiento es semejante a cambios en el radio $a$ de las NPs.  Este análisis se llevó a cabo dado que tanto el radio $a$ como la fracción de cubierta $\Theta$ modifican el volumen neto de material plasmónico, es decir, hay en la monocapa una mayor cantidad de electrones libres, así como la distancia promedio $\langle d  \rangle$ entre las NPs. Si  los mínimos en $R$  a energías menores que la de la SP-SPR dipolar son sensibles al radio de las NPs, como lo es con la fracción de cubierta, se corrobora que esta excitación se debe a un efecto colectivo de las NPs, puesto que se ve afectada por la cantidad de electrones libres dentro de la monocapa.
 
Los resultados de la reflectancia de una monocapa con $\Theta=0.3$, inmersa en aire ($n_m = 1$) y soportada por un sustrato con índice de refracción $n_m= 1.5$, se muestran en la Fig.  \ref{fig:R-RVar}, como función del ángulo de incidencia, tanto de la longitud de onda $\lambda$ (escala inferior) como de la  energía $\hbar\omega$ (escala superior) del haz incidente. Se consideraron NPs  con una respuesta EM dada dada por una función dieléctrica  tipo Drude [Ec. \eqref{eq:Drude}] con los parámetros $\hbar\omega_p =4.3$ eV y $\hbar\gamma=0.15$ eV, cuyos radios $a$ fueran los siguientes: $3$ nm, $5$ nm, $10$ nm y $20$ nm. La reflectancia en polarización \emph{p} se presenta en las gráficas $\mathbf{i)-iv)}$,mientras que en \emph{s}, en las gráficas $\mathbf{v)-viii)}$. Las SP-SPRs dipolar y cuadrupolar corresponden a las líneas punteadas verde y rosa, respectivamente. Para $a = 3$ nm y $5$ nm la excitación dipolar se localiza en $\lambda\approx 500$ nm, para el radio  $a = 10$ nm en $\lambda\approx 503$ nm y $a=20$ nm en $\lambda\approx 512$ nm, mientras que la SP-SPR cuadrupolar se localiza en $456$ nm para $a\leq 10$ nm y para el caso  $a=20$ nm, $\lambda = 462$ nm.

	\begin{figure}[h!]\centering
\includegraphics[width = .76\linewidth]{2-Resultados/figs/3-Wp4rVar/0-2D_Grid}%
\includegraphics[scale=.89, trim={00 -5 00 00}, clip]{2-Resultados/figs/0-RBar_v}\vspace*{-.5em}
	\caption{Gráficas de la reflectancia de una monocapa en configuración ATR como función del ángulo de incidencia $\theta_i$ y de la longitud de onda $\lambda$ (escala inferior), así como de la energía del haz incidente en unidades de $\hbar\omega$ (escala superior), para una función dieléctrica tipo Drude con $\hbar\omega_p=4.3$ eV  y  $\hbar\gamma=0. 15$ eV.  Las gráficas   en el renglón superior [$\mathbf{i)-v)}$] muestran los resultados para  polarización \emph{p} y las del renglón inferior  [$\mathbf{vi)-x)}$]  para polarización  \emph{s}, donde se consideró una fracción de cubierta $\Theta = 0.3$ y  NPs de radio  $a$: $3$ nm, $5$ nm, $10$ nm y $20$ nm.  Las líneas verticales punteadas verdes y rosas corresponden a las SP-SPRs dipolar y  cuadrupolar, respectivamente.  Los puntos amarillos corresponden a los mínimos en $R$ para ángulos mayores a $\theta_c\approx 41.8^\circ$ y longitudes de onda mayores a la SP-SPRs dipolar.
}	\label{fig:R-RVar}	
	\end{figure}	

En la Fig.   \ref{fig:R-RVar} (variación de $a$) la respuesta EM de la monocapa es análoga al de la Fig. \ref{fig:R-ATR4} (variación de $\Theta$) puesto que se presentan  dos regiones en $\theta_i>\theta_c=41.8^\circ$  donde se cumple que $R<1$: en $\lambda$ cercanas a las SP-SPRs y en longitudes de onda mayores a la excitación dipolar de una partícula. La distancia entre estas regiones aumenta al crecer el radio de las NPs, al igual que lo hacía al aumentar la fracción de cubierta, ademas de que esta distancia es mayor para polarización \emph{p} que para \emph{s}. Dado que la excitación a energías menores a las de las SP-SPRs (puntos amarillos) se modifica al aumentar el radio de las NPs, y no solo al cambiar el valor de la fracción de cubierta, esta excitación corresponde a un modo coletivo ya que responde a la cantidad neta de material plasmónico ---es decir, de electrones libres--- presentes en la monocapa. Para analizar la respuesta EM de la monocapa al aumentar el radio de las NPs, y compararla con la variación en $\Theta$ en la Fig. \ref{fig:R-ATR4},  se grafica la reflectancia a $\theta_i = 65^\circ$. 

En la Fig. \ref{fig:R-RVar-Cuts} se presentan cortes de la reflectancia graficada en la Fig. \ref{fig:R-RVar} a $\theta_i = 65^\circ$. Dado que la longitud de onda de las SP-SPRs depende del radio de las NPs, la excitación dipolar para los tamaños de partículas utilizadas corresponde a la región verde entre $500$ nm y $512$ nm, mientras que la cuadrupolar corresponde a la región rosa entre $456$ nm y $462$ nm.
 En los resultados de la reflectancia para polarización \emph{p}, graficados en la Fig. \ref{sfig:R-RVar-cutp}, la excitación cuadrupolar sólo es apreciable para $a=20$ nm, y la SP-SPR dipolar se corre al rojo para $a\geq 5$ nm; las excitaciones a $\lambda$ mayores de $512$ nm se atribuyen a la respuesta colectiva, apreciable para todos los radios considerados. Para polarización \emph{s}, Fig. \ref{sfig:R-RVar-cuts}, la respuesta cuadrupolar sólo se observa para $a = 20$ nm y en ningún caso se observa un corrimiento al azul de la SP-SPR dipolar. Los mínimos de la reflectancia dentro del rango de la SP-SPR dipolar se corren al rojo conforme crece el radio y la disminución en el valor de $R$ es mucho menor que la disminución  observada en la Fig. \ref{sfig:R-ATR4-cutp} (respuesta EM de la monocapa de NPs tipo Drude con $\hbar\omega_p = 4.3$, $a = 30$ nm y variaciones en $\Theta$) para la SP-SPR dipolar. Las excitaciones a $\lambda>500$ nm en la Fig. \ref{fig:R-RVar-Cuts} siguen las tendencias observadas en el modo colectivo, se atribuyen a éste y se corrobora que la excitación colectiva se traslapa con la SP-SPR dipolar, pues para los radios $a\leq 10$ nm  y polarización \emph{s} el término dipolar es el dominante (considerando una sola partícula) sin embargo, no se aprecia en la reflectancia de la monocapa, mientras que para polarización \emph{p} esto es válido únicamente para $a = 3$ nm, en donde el desplazamiento del modo colectivo respecto a la SP-SPR dipolar es menor; para radios mayores la separación entre ambas excitaciones es mayor, lo que permite identificarlas entre sí. 
 
\begin{figure}[h!]\centering\hspace*{-1.5em}
	\begin{subfigure}{.01\linewidth}\caption{}\label{sfig:R-RVar-cutp}\vspace{4.5cm}\end{subfigure}
	\begin{subfigure}{.45\linewidth}\hspace*{-1.5em}
	\includegraphics[scale=1]{2-Resultados/figs/3-Wp4rVar/cut_angle_65_p.pdf}\end{subfigure}
	\begin{subfigure}{.01\linewidth}\caption{}\label{sfig:R-RVar-cuts}\vspace{4.5cm}\end{subfigure}\hspace*{-1.em}
	\begin{subfigure}{.45\linewidth}\centering
	\includegraphics[scale=1 ]{2-Resultados/figs/3-Wp4rVar/cut_angle_65_s.pdf}\end{subfigure}\vspace*{-.5em}
	\caption{Cortes de la Fig. \ref{fig:R-RVar} a $\theta_i = 65^\circ$ de las gráficas de reflectancia de una monocapa en configuración ATR (Fig. \ref{fig:R-RVar}) de NPs esféricas de fracción de cubierta $\Theta = 0.3$ en polarización \textbf{a)} \emph{p} y \textbf{b)} \emph{s} en función de la longitud de onda $\lambda$ (escala inferior) y de la energía $\hbar\omega$ (escala superior). Los parámetros de la función dieléctrica tipo Drude para las NPs son $\hbar\omega_p = 4.3$ eV y $\hbar\gamma = 0.15$ eV y las fracciones de cubierta consideradas fueron $a$: $3$ nm, $5$ nm, $10$ nm y $20$ nm. La SP-SPR dipolar para los tamaños de partículas utilizadas corresponde la región verde entre $500$ nm y $512$ nm, mientras que la cuadrupolar corresponde a la región rosa entre $456$ nm y $462$ nm.}\label{fig:R-RVar-Cuts}
	\end{figure}	
{\color{red}
NO SÉ SI SE HARÁ UNA TABA O UNA GRÁFICA. COMENTAR CON ALEJANDRO :(
Con base en las Figs. \ref{fig:R-ATR4-Cuts}, y \ref{fig:R-RVar-Cuts} (reflectancia a $\theta_i=65^\circ$  de una monocapa de NPs variando la fracción de cubierta de la monocapa y el radio de las NPs, respectivamente; en ambos casos considerando $\hbar\omega_p = 4.3$ eV para la función dieléctrica de las NPs), el modo colectivo se excita a una longitud de onda  $\lambda_{exc}$ y la excitación se corre al rojo respecto a la SP-SPR dipolar al aumentar el radio $a$ de las NPs o la fracción de cubierta $\Theta$ de la monocapa, por lo que es posible sintonizar la longitud de onda a la que se excita mediante estos parámetros. Sin embargo, el valor de la reflectancia $R$ en $\lambda_{exc}$ y el ancho de la excitación $\Delta\lambda_{FWHM}$ también se modifican al cambiar el valor de  $a$ y $\Theta$. Para determinar los valores de $a$ y $\Theta$ óptimos en el biosensado, se presentan en la Tab. \ref{tab:ExcColectiva} la longitud de onda de excitación $\lambda_{exc}$ del modo colectivo observado en las Figs.  \ref{fig:R-ATR4-Cuts}, y \ref{fig:R-RVar-Cuts}, así como su desplazamiento respecto a la SP-SPRR dipolar $\Delta\lambda^{(1)} = \lambda_{exc}-\lambda^{(1)}$, en donde $\lambda^{(1)}$ corresponde a la longitud de onda de la SP-SPR dipolar ($\ell = 1$), el ancho de la excitación colectiva $\Delta\lambda_{FWHM}$ y el valor de la reflectancia en $\lambda_{exc}$. 

\begin{table}[h]
\caption{Longitudes de onda de excitación $\lambda_{exc}$ del modo colectivo, su desplazamiento respecto a las SP-SPRs dipolares $\Delta\lambda^{(1)}$, el ancho de la excitación colectiva $\Delta\lambda_{FWHM}$ y el valor de la reflectancia $R_p$ y $R_s$ de la excitación para polarización \emph{p} y \emph{s}, respectivamente. Se consideró una monocapa de NPs  con un índice de refracción dado porl modelo de Drude-Sommerfeld con los parámetros $\hbar\omega_p = 4.3$ eV y $\hbar\gamma = 0.15$ eV, además de un ángulo de incidencia $\theta_i = 65^\circ$. En la tabla se presentan los casos de una monocapa de NPs con radio $a=30$ nm y $\Theta = 0.05,\,0.1,\,0.2,\,0.3$ y $0.4$, correspondientes a la Fig.  \ref{fig:R-ATR4-Cuts}, así como una monocapa con $\Theta = 0.3$ y NPs con radios $a = 3$ nm, $5$ nm, $10$ nm, $20$ nm y $30$ nm,  correspondientes a la Fig.  \ref{fig:R-RVar-Cuts}. Para $a = 3$ nm y $5$ nm la excitación dipolar se localiza en $\lambda^{(1)}\approx 500$ nm, para el radio  $a = 10$ nm en $\lambda^{(1)}\approx 503$ nm y $a=20$ nm en $\lambda^{(1)}\approx 512$ nm}
\label{tab:ExcColectiva}
\centering
\begin{tabular}{c|c|c||c|c|c|c|c}
\cline{1-8}\hline\hline 
\multicolumn{2}{c}{Fig. \ref{fig:R-ATR4-Cuts} } & $\Theta$ & $0.05$ & $0.1$ & $0.2$ & $0.3$ & $0.4$ \\ \hline\hline
\multirow{8}{*}{$a = 30 $ nm} & \multirow{4}{*}{Pol. \emph{p}} & $\lambda_{exc}$ [nm] & $554$ & $585$ & $654$ & $731$ & $816$ \\ 
&  & $\Delta\lambda^{(1)}$ [nm] & $28$ & $60$ & $128$ & $205$ & $290$ \\ 
 &  & $\Delta\lambda_{FWHM}$ [nm] & $76$ & $118$ & $136$ & $198$ & $275$ \\ 
 &  & $R_p$ & $0.25$ & $0.036$ & $0.043$ & $0.19$ & $0.33$ \\ \cline{2-8}
 & \multirow{4}{*}{Pol. \emph{s}} & $\lambda_{exc}$ [nm] & $538$ & $551$ & $576$ & $603$ & $630$ \\ 
 &  & $\Delta\lambda^{(1)}$ [nm] & $12$ & $25$ & $50$ & $77$ & $104$ \\ 
 &  & $\Delta\lambda_{FWHM}$ [nm] & $57$ & $69$ & $94$ & $123$ & $155$ \\ 
 &  & $R_s$ & $0.41$ & $0.14$ & $7.2\times 10^{-4}$ & $0.04$ & $0.12$ \\
 \hline \hline\hline 
\multicolumn{2}{c}{ Fig. \ref{fig:R-RVar-Cuts}}  & $a$ [nm] & $3$ & $5$ & $10$ & $20$ & $30$ \\  \hline \hline
\multirow{8}{*}{$\Theta = 0.3$} & \multirow{4}{*}{Pol. \emph{p}} & $\lambda_{exc}$ [nm] & $522$ & $537$ & $573$ & $650$ & $731$ \\
 &  & $\Delta\lambda^{(1)}$ [nm] & $22$ & $37$ & $71$ & $138$ & $205$ \\ 
 &  & $\Delta\lambda_{FWHM}$ [nm] & $56$ & $76$ & $95$ & $144$ & $198$ \\ 
 &  & $R_p$ & $0.09$ & $0.009$ & $0.03$ & $0.14$ & $0.19$ \\ \cline{2-8}
 & \multirow{4}{*}{Pol. \emph{s}} & $\lambda_{exc}$ [nm] & $508$ & $514$ & $530$ & $564$ & $603$ \\ 
 &  & $\Delta\lambda^{(1)}$ [nm] & $52$ & $27$ & $14$ & $8$ & $77$ \\ 
 &  & $\Delta\lambda_{FWHM}$ [nm] & $42$ & $48$ & $65$ & $94$ & $123$ \\ 
 &  & $R_s$ & $0.22$ & $0.07$ & $2.7\times 10^{-6}$ & $0.03$ & $0.04$\\  \hline\hline \cline{1-8}
\end{tabular}
\end{table}

En el análisis donde se empleo una función dieléctrica tipo Drude con el parámetro $\hbar\omega_p= 4.3$ eV para una monocapa de NPs con radio $a=30$ nm y se analizó la reflectancia de ésta a $\theta_i = 65^\circ$ en función de $\lambda$ para distintos valores de la fracción de cubierta (ver Fig. \ref{fig:R-ATR4-Cuts}) se observo que el modo colectivo se excita a $\lambda = 830$ nm a polarización \emph{p} y a $\lambda = 640$ nm en \emph{s}, es decir, que se corren al rojo respecto a la SP-SPR dipolar $320$ nm y $120$ nm,   respectivamente.
}

La respuesta óptica de la monocapa correspondiente al modo colectivo estudiado en las Figs. {\color{red} poner las fig que quiero hacer}, en donde se presenta el corrimeinto al rojo respecto a la SP-SPR dipolar y el ancho de la excitación colectiva para las variaciones de la fracción de cubierta, así como a variaciones del radio, tienen un comportamiento semejante. A mayor cantidad de electrones libres, mayor el corrimiento al rojo; respuesta más evidente para polarización \emph{p} que \emph{s}, así como el valor mínimo en $R$ a la longitud de onda de la excitación colectiva  para valores medios de la cantidad de electrones libres en el material. Dado que el modo colectivo estudiado se presenta a energías menores a la de las SP-SPRs, ésta no es una excitación plasmónica, de forma semejante que las PSLRs reportadas en  \cite{kabashin2009plasmonic} y \cite{danilov2018ultra}. Sin embargo, las PSLRs son una excitación colectiva que se presenta en un arreglo ordenado de nanocilindros \cite{kabashin2009plasmonic} y \cite{danilov2018ultra} que, además, cuenta con las características de un modo guiado, es decir, que la energía se propaga a través de la monocapa de nanocilindros \cite{kabashin2009plasmonic}. El modo colectivo  para sistemas desordenados de NPs esféricas idénticas (puntos amarillos en las Figs. \ref{fig:R-ATR4}, \ref{fig:R-ATR10} y \ref{fig:R-RVar}) se caracterizó mediante los mínimos en la reflectancia a las  longitudes de onda de excitación $\lambda_{exc}$, mayores a las de las SP-SPRs. Si la transmitancia $T$ de la monocapa evaluada en $\lambda_{exc}$ no es máxima, entonces el modo colectivo presenta características de un modo guiadp, semejante a las PSLRs.

 En la Fig. \ref{fig:RT-Omegas} se muestran los cálculos de la reflectancia $R$, la transmitancia $T$ y la suma de éstas ($R+T$) en función del ángulo de incidencia $\theta_i$, así como de la longitud de onda $\lambda$ (escala inferior) y de la energía del haz incidente $\hbar\omega$ (escala superior), tanto para polarización \emph{p}  [\textbf{i)}--\textbf{iii)}] como para \emph{s} [\textbf{iv)}--\textbf{vi)}], de una monocapa de NPs esféricas de radio $a = 30$ nm inmersa en una matriz de aire ($n_m = 1$) y soportada sobre un sustrato con un índice de refracción $n_s = 1.5$; la función dieléctrica de las NPs que conforman a la monocapa está descrita por el modelo de Drude-Sommelfeld con los parámetros $\hbar\omega_p = 4.3$ eV y $\hbar\gamma = 0.15$ eV en la Fig. \ref{sfig:RT-4}, y $\hbar\omega_p = 10$ eV y $\hbar\gamma=0.15$ eV en la Fig. \ref{sfig:RT-10}. Cuando se emplea el parámetro $\hbar\omega_p = 4.3$ eV [Fig. \ref{sfig:RT-4}], la SP-SPR dipolar  (línea punteada vertical verde) se encuentra en $526$ nm, mientras que la cuadrupolar (línea punteada vertical rosa)  en $462$ nm. En el caso de $\hbar\omega_p=10$ eV, se observa, además de  la SP-SPR dipolar en $265$ nm y la cuadrupolar en $211$ nm, la resonancia octopolar en $195$ nm, que corresponde a la línea punteada vertical cian en la Fig. \ref{sfig:RT-10}.

\begin{figure}[h!]\centering
	\begin{subfigure}{.01\linewidth}\caption{}\label{sfig:RT-4}\vspace{6.5cm}\end{subfigure}
	\begin{subfigure}{.7\linewidth}\hspace*{-.5em}
	\includegraphics[scale=.58]{2-Resultados/figs/5-RT-Wp4-10/0-2D_Grid_1.png}%	
	\includegraphics[scale=.89, trim={00 -5 00 00}, clip]{2-Resultados/figs/0-IBar_v}
	\end{subfigure}\\
	\begin{subfigure}{.01\linewidth}\caption{}\label{sfig:RT-10}\vspace{6.5cm}\end{subfigure}\hspace*{-1em}
	\begin{subfigure}{.7\linewidth}\centering
	\includegraphics[scale=.58 ]{2-Resultados/figs/5-RT-Wp4-10/0-2D_Grid_2.png}%
		\includegraphics[scale=.89, trim={00 -5 00 00}, clip]{2-Resultados/figs/0-IBar_v}
		\end{subfigure}\vspace*{-.5em}
	\caption{Gráficas de reflectancia $R$, transmitancia $T$ y la suma de éstas $R+T$ de una monocapa en configuración ATR como función del ángulo de incidencia $\theta_i$ y de la longitud de onda $\lambda$ (escala inferior) así como de la energía del haz incidente en unidades de $\hbar\omega$ (escala superior), para una función dieléctrica tipo Drude con \textbf{a)} $\hbar\omega_p=4. 3$ eV  y  $\hbar\gamma=0. 15$ eV y \textbf{b)} $\hbar\omega_p = 10$ eV y $\hbar\gamma = 0.15$ eV.  Las gráficas   en el renglón superior [$\mathbf{i)-ii)}$]  muestran los resultados de reflectancia para  polarización \emph{p} y las del renglón inferior  [$\mathbf{iv)-vi)}$] para polarización  \emph{s}, donde se consideraron NPs de radio $a=30$ nm. Las líneas verticales punteadas verdes corresponden a la SP-SPRs dipolar ($526$ nm y $265$ nm para $\hbar\omega_p=4.3$ eV y $\hbar\omega_p = 10$ eV, respectivamente), las rosas a la SP-SPR cuadrupolar ($462$ nm y $211$ nm para $\hbar\omega_p=4.3$ eV y $\hbar\omega_p = 10$ eV, respectivamente) y las cianes a la SP-SPR octopolar ($195$ nm para $\hbar\omega_p = 10$ eV). Los puntos amarillos corresponden a los mínimos en $R$, y $R+T$ para ángulos mayores a $\theta_c\approx 41.8^\circ$ y longitudes de onda mayores a la SP-SPRs dipolar. }\label{fig:RT-Omegas}
	\end{figure}	

Para ambos casos analizados en la Fig. \ref{fig:RT-Omegas}, $\hbar\omega_p = 4.3$ eV y $\hbar\omega_p = 10$ eV, se observa que para valores de $\lambda$ cercanos a los de las SP-SPRs (líneas punteadas verticales) la reflectancia $R$ presenta máximos locales para ángulo de incidencia $\theta_i<\theta_c \approx 41.8^\circ$ y mínimos locales para $\theta_i>\theta_c$. De forma contraria, la transmitancia $T$ es cercana a cero para todo $\theta_i$ en los valores de $\lambda$ cercanos a los de las SP-SPRs. La suma de $R$ y $T$ [\ref{fig:RT-Omegas} \textbf{iii)} y \textbf{vi)}] en los valores de $\lambda$ que corresponden a las SP-SPRs, es menor a la unidad debido a que en esta región las NPs, en el límite de partícula individual, extinguen la luz de forma eficiente, en donde la absorción contribuye más que el esparcimiento {\color{red} poner graficas de extinción de luz para 30nm}. Sin embargo, las NPs no absorben en longitudes de onda mayores a la de la SP-SPR dipolar (líneas punteadas verdes) por lo que la extinción de luz en el presunto modo guiado (puntos amarillos en $R$ y $R+T$) se debe al esparcimiento de los campos EMs debido a la interacción de la onda evanescente que incide sobre las NPs y éstas. Dado que la transmitancia es igual a cero en los valores de $\theta_i$ y $\lambda$ correspondientes al presunto modo colectivo, el esparcimiento de luz no ocurre en la dirección transmitida coherente, como tampoco a la reflejada. 

El estudio de la excitación distinta de las SP-SPRs presente en los cálculos de la reflectancia de una monocapa de NPs esféricas e idénticas, permite catalogarlo como un modo colectivo dado que su respuesta es más apreciable al aumentar la fracción de cubierta $\Theta$ y el radio $a$ de las NPs. Asimismo, al analizar la la extinción de luz  en los cálculos de $R$ y $T$, es posible catalogar a la excitación atípica presente en la monocapa como un modo guiado puesto que se presenta a longitudes de onda en donde las NPs no absorben y el esparcimiento de luz no corresponde a la componente coherente y, por hipótesis en la construcción del CSM, la componente difusa es despreciable respecto a ésta. Tras la caracterización de la excitación atípica como un modo guiado y colectivo, que es sintonizable según los parámetros de la monocapa, empleando el modelo de Drude-Sommerfeld para la función dieléctrica de las NPs, se analiza si esta excitación colectiva es apreciable en materiales más realistas. En la siguiente sección se presentan los resultados de la respuesta EM de una monocapa conformado por NPs de oro y plata, es decir, empleando como función dieléctrica de las NPs la corrección por tamaño de los datos experimentales para el oro y plata en bulto de \cite{johnson1972constants}.

