% !TeX root = ../tesis.tex
\chapter*{Conclusiones}\addcontentsline{toc}{chapter}{\protect\numberline{}Conclusiones}
\label{chapter:Conclusiones}

El objetivo principal de esta tesis de licenciatura fue carcaterizar la respuesta electromagnética de una monocapa desordenada de nanopartículas esféricas para proponer su uso en biosensado. Se calculó la  reflectancia y transmitancia de la monocapa cuando una onda plana monocromática linealmente polarizada incide sobre ésta, empleando el modelo de esparcimiento coherente \cite{reyes2018analytical,pena-gomar2006coherent,barrera1991optical,garcia2012multiple} ---el cual proporciona expresiones analíticas de la reflectancia y transmitancia cuando la monocapa se encuentra embebida en un dieléctrico, denominado matriz, y soportada por un sustrato--- y considerando para  la función dieléctrica de las nanopartículas que forman la monocapa tanto el modelo de Drude-Sommerfeld como  la corrección por tamaño de la función dieléctrica del oro y de la plata para nanopartículas esféricas. Asimismo, se comparó la sensibilidad de la respuesta electromagnética de una monocapa de nanopartículas esféricas de oro ante cambios en el índice de refracción de la matriz con la de propuestas de biosensores nanoestructurados reportados en la literatura \cite{svedendahl2009refractometric,kabashin2009plasmonic,danilov2018ultra}, basados en la resonancia de plasmón de superficie localizada y en la resonancia de red de superfice plasmónica. %Finalmente, se calculó la figura de mérito de bulto $\textit{FoM}_B$ para el supuesto modo colectivo y su longitud de penetración.

Al emplear el modelo de esparcimiento coherente considerando nanopartículas con una función dieléctrica tipo Drude, se identificó un supuesto modo plasmónico colectivo que puede sintonizarse seleccionando el radio $a$ de las nanopartículas de la monocapa y su fracción de cubierta $\Theta$. El supuesto modo plasmónico colectivo se excita en un esquema de reflexión total atenuada y a energías menores a las del plasmón dipolar de superficie localizado de las nanopartículas individuales formando la monocapa, además de presentarse tanto para polarización \emph{p} como \emph{s}. Adicionalmente,  al analizar la reflectancia y transmitancia de la monocapa a las longitudes de onda del supuesto modo plasmónico colectivo, se observó que éste presenta características de un modo guiado. Al considerar materiales reales (oro y plata) para las nanopartículas en la monocapa también se observó la presencia del supuesto modo plasmónico colectivo. Al analizar la respuesta electromagnética de una  monocapa desordenada de nanopartículas esféricas ante cambios en el índice de refracción de la matriz $n_m$, se observó que la sensibilidad de bulto $S_B$ del supuesto  modo plasmónico colectivo (considerando una monocapa con nanopartículas de oro con  $a=30$ nm y $\Theta=0.125$) es semejante a la de los arreglos nanoestructurados desordenados de nanodiscos y ordenados de nanocilindros, reportados en \cite{svedendahl2009refractometric} y \cite{danilov2018ultra}, respectivamente. Por ejemplo, la sensibilidad del supuesto modo plasmónico colectivo es mayor para $\theta_i = 70^\circ$ y a ambas polarizaciones cuando $1.35 \leq n_m \leq 1.39$ (comparado a los nanodiscos), y para $\theta_i=73^\circ$ a ambas polarizaciones cuando $1.36\leq n_m \leq 1.42$ (comparado a los nanocilindros). La figura de mérito de bulto ---la sensibilidad de bulto dividida por el ancho de la resonancia (Full Width at Half Maximum, FWHM)--- para el supuesto modo plasmónico colectivo es consistente con el reportado en la literatura para arreglos nanoestructurados \cite{svedendahl2009refractometric}. Al comparar la respuesta electromagnética del supuesto modo plasmónico colectivo con la del plasmón-polaritón de superficie (excitado en una película continua de oro de $50$ nm de grosor), el primero tiene un sensiblidad menor mas, dependiendo del intervalo de $n_m$, la sensibilidad de estas excitaciones es del mismo orden de magnitud. Cabe resaltar que el supuesto modo plasmónico colectivo se excita a ambas polarizaciones ---en contraste con el plasmón-polaritón de superficie que sólo se excita en polarización \emph{p}--- y presenta un corrimiento de su longitud de onda de excitación tanto hacia al rojo como al azul al aumentar el índice de refracción de la matriz ---característica no observada para la excitación de la película continua de oro, ni para los arreglos nanoestructurados de nanocilindros ni nanodiscos---.

 Como resultado de este trabajo de tesis, se determinó que una monocapa desordenada de nanopartículas esféricas, e idénticas, de oro con $a=30$ nm y  $\Theta=0.125$ y una de nanopartículas de plata con $a=40$ nm y $\Theta=0.1$ pueden emplearse para biosensado, ya que estos parámetros sintonizan al supuesto modo plasmónico colectivo dentro del espectro visible y minimizan la reflectancia a las longitudes de onda de este modo colectivo a ángulos de incidencia menores a $80^\circ$, facilitando su medición. 
 
Como continuación al trabajo presentado en esta tesis, se propone la identificación experimental del supuesto modo plasmónico colectivo con monocapas desordenadas de nanopartículas esféricas de oro y de plata con los valores de $a$ y $\Theta$ propuestos para cada material. La propuesta para la fabricación de las nanopartículas esféricas de oro y de plata es sintetizar las nanopartículas, en una suspensión colidal, mediante el método de Turkevich \cite{wuithschick2015turkevich}, para luego depositar las nanopartículas sobre un sustrato en concentraciones bajas, formando la monocapa. Para la detección experimental del supuesto modo plasmónico colectivo, se propone realizar la medición de la reflectancia y la transmitancia, dentro del espectro visible,  para un ángulo de incidencia entre el ángulo crítico y $80^\circ$. Asimismo, se propone calcular la figura de mérito de superficie, que cuantifica la sensibilidad ante cambios de índice de refracción locales al rededor de los elementos de arreglos nanoestructurados \cite{estevez2014trends,svedendahl2009refractometric}, considerando un analito particular para funcionalizar a las nanopartículas de la monocapa y estudiar el corrimiento de la longitud de onda de excitación del supuesto modo plasmónico colectivo cuando se forma una capa uniforme al rededor de las nanopartículas.


















