% !TeX root = ../tesis.tex

% Thesis Abstract -----------------------------------------------------

%\begin{abstractslong}    %uncommenting this line, gives a different abstract heading
\begin{abstracts}        %this creates the heading for the abstract page
\addcontentsline{toc}{chapter}{\protect\numberline{}Resumen}
\vspace*{-.75cm}
\small
Los biosensores plasmónicos comerciales tradicionalmente emplean plasmones-polaritones de superficie excitados en una película continua de oro. Recientemente, se han propuesto arreglos nanoestructurados tanto periódicos como desordenados para el biosensado empleando resonancias plasmónicas de red de superficie  y  resonancias de plasmón de superficie localizadas,  respectivamente. Sin embargo, en general, su fabricación es lenta y costosa, pues involucran técnicas de fabricación como la litografía coloidal de hueco-máscara, y deposición por evaporación catódica y  formación de poros. Como una alternativa, en este trabajo de tesis de licenciatura se estudia la posibilidad de emplear sistemas monocapa desordenados de nanopartículas esféricas tanto de oro como de plata para biosensado, analizando su reflectancia y transmitancia de forma teórica por medio del modelo de esparcimiento coherente. Dicho modelo predice la existencia de un supuesto modo  plasmónico colectivo de la monocapa, excitado en incidencia interna, que puede sintonizarse al elegir el radio $a$ de las nanopartículas que forman la monocapa y su fracción de cubierta $\Theta$. Los resultados obtenidos muestran que una monocapa desordenada de nanopartículas esféricas de oro con radio $a=30$ nm y $\Theta=0.125$ podría emplearse para el biosensado al igual que una monocapa de nanopartículas de plata de $a=40$ nm y $\Theta=0.1$. La comparación del supuesto modo plasmónico colectivo predicho por el modelo de esparcimiento coherente con la resonancia plasmónica de red de superficie y la resonancia de plasmón de superficie localizada muestra sensibilidades de bulto similares para ángulos de incidencia cercanos al ángulo crítico. Adicionalmente, en este trabajo se compara la sensibilidad del supuesto modo plasmónico colectivo ---excitable en ambas polarizaciones--- con la del plasmón-polaritón de superficie ---que sólo se excita en polarización \emph{p}---,  mostrando que existen condiciones particulares en las que la sensibilidad de una monocapa de nanopartículas de oro es comparable con la de una película delgada de oro, resultado también observado al emplear plata.

\vspace*{.5cm}
\textit{
Commercial plasmonic biosensors use traditionally surface plasmon-polaritons excited on a continuous gold film. Periodic and disordered nanostructured arrays have been recently proposed for biosensing by employing plasmonic surface lattice resonances and localized surface plasmon resonances. However their production is generally expensive and time consuming due to their fabrication techniques such as hole-mask colloidal lithography or sputtering deposition followed by pore formation. As an alternative, in this bachelor thesis it is studied the possibility to use disordered monolayers of gold and silver spherical nanoparticles as a biosensor, by analyzing theoretically their reflectance and transmittance by means of the coherent scattering model. This model predicts a collective-like plasmonic mode, excited in an internal reflectance configuration and for both polarizations, which can be tuned by choosing the radius ($a$) of the nanoparticles in the monolayer and its cover fraction $\Theta$. The obtained results show that a monolayer of gold nanoparticles with $ a = 30 $ nm and $ \Theta = 0.125 $  could be used for biosensing, as well as a monolayer of silver nanoparticles with  $a=40$ nm and $\Theta=0.1$. The performed comparison for the bulk sensitivity between the collective-like plasmonic mode and both the plasmonic surface lattice resonances and the localized surface plasmon resonance shows it is similar for angles of incidence close to the critical angle. In addition, the sensitivity of the collective-like plasmonic mode and the surface plasmon-polariton were compared, showing that under particular conditions the sensitivity of a monolayer formed by gold nanoparticles is comparable to that of a thin gold film; this result is also observed when silver is used. }

\end{abstracts}
%\end{abstractlongs}


% ----------------------------------------------------------------------