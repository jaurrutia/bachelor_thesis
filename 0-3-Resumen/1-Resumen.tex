
% Thesis Abstract -----------------------------------------------------

%\begin{abstractslong}    %uncommenting this line, gives a different abstract heading
\begin{abstracts}        %this creates the heading for the abstract page
\addcontentsline{toc}{chapter}{\protect\numberline{}Resumen}
\vspace{-1em}

Los biosensores plasmónicos comerciales emplean plasmones-polaritones de superfice (SPPs) excitados en una película continua de oro. Se han propuesto arreglos nanoestructurados tanto periódicos como desordenados para mejorar la sensibilidad de los sensores comerciales empleando las resonancias plasmónicas de red de superficie (PSLRs) y  las resonancias de plasmón de superficie localizadas (LSPRs),  respectivamente, sin embargo, su fabricación es costosa y tardada. Con el modelo de esparcimiento coherente (CSM) se analiza de forma teórica la reflectancia de una monocapa desordenada de nanopartículas (NPs) esféricas de oro y de plata y se evalúa su uso como sensor. El CSM predice un supuesto modo plasmónico colectivo, excitado en incidencia interna, que puede sintonizarse al elegir el radio $a$ de las NPs de la monocapa y su fracción de llenado $\Theta$. Los resultados con el CSM muestran que una monocapa desordenada de NPs de oro con radio $a=30$ nm y $\Theta=0.125$ y una de NPs de plata con $a=40$ nm y $\Theta=0.1$ pueden emplearse para el biosensado. La comparación del supuesto modo colectivo con la PSLR y la LSPR muestra sensibilidades comparables entre las tres resonancias para ángulos de incidencia cercanos al ángulo crítico. La comparación del supuesto modo plasmónico colectivo con el SPP muestra longitudes de penetración comparables a pesar de que el SPP cuenta con una sensibilidad y una figura de mérito de bulto mayores.

\vspace*{1cm}\textit{
Commercial plasmonic biosensors use surface plasmon-polaritons (SPPs) excited on a continuous gold film. Periodic and disordered nanostructured arrays have been proposed to improve the sensitivity of commercial sensors by employing Plasmonic Surface Lattice Resonances (PSLRs) and Localized Surface Plasmon Resonances (LSPRs), however their production time is long and their cost is expensive.  By using the coherent spreading model (CSM) the reflectance of a monolayer of spherical gold and silver nanoparticles (NPs) randomly arranged is theoretically analyzed and its use as a sensor is evaluated. The CSM predicts a possible collective plasmonic mode, excited in internal incidence, which can be tuned by choosing the radius $a$ of the NPs on the monolayer and its cover fraccion $\Theta$. The results with the CSM show that a monolayer of gold NPs with radius $ a = 30 $ nm and $ \Theta = 0.125 $, as well as one of silver NPs with $ a = 40 $ nm and $ \Theta = 0.1 $  can be used for biosensing. The comparison of the possible collective plasmonic mode against the PSLR and the LSPR shows comparable sensitivities between the three resonances for angles of incidence close to the critical angle. The comparison of the probable collective plasmonic mode with the SPP shows comparable penetration lengths even though the SPP has both a greater sensitivity and a greater figure of merit.}



\end{abstracts}
%\end{abstractlongs}


% ----------------------------------------------------------------------