
% Thesis Abstract -----------------------------------------------------

%\begin{abstractslong}    %uncommenting this line, gives a different abstract heading
\begin{abstracts}        %this creates the heading for the abstract page
\addcontentsline{toc}{chapter}{\protect\numberline{}Resumen}


Los biosensores plasmónicos comerciales emplean plasmones-polaritones de superficie excitados en una película continua de oro. Se han propuesto arreglos nanoestructurados tanto periódicos como desordenados para el biosensado empleando las resonancias plasmónicas de red de superficie  y  las resonancias de plasmón de superficie localizadas,  respectivamente, sin embargo, su fabricación es costosa y tardada. Con el modelo de esparcimiento coherente se analiza de forma teórica la reflectancia de una monocapa desordenada de nanopartículas esféricas de oro y de plata y se evalúa su uso como sensor. El modelo de esparcimiento coherente predice un supuesto modo plasmónico colectivo, excitado en incidencia interna, que puede sintonizarse al elegir el radio $a$ de las NPs de la monocapa y su fracción de cubierta $\Theta$. Los resultados obtenidos muestran que una monocapa desordenada de NPs de oro con radio $a=30$ nm y $\Theta=0.125$ pueden emplearse para el biosensado. La comparación del supuesto modo plasmónico colectivo con la resonancia plasmónica de red de superficie y la resonancia de plasmón de superficie localizada muestra sensibilidades comparables entre las tres resonancias para ángulos de incidencia cercanos al ángulo crítico. 

\vspace*{1cm}\textit{
Commercial plasmonic biosensors use surface plasmon-polaritons excited on a continuous gold film. Periodic and disordered nanostructured arrays have been proposed for biosensing by employing plasmonic surface lattice resonances and localized surface plasmon resonances, however their production time is long and their cost is expensive.  By using the coherent scattering model the reflectance of a  monolayer of spherical gold and silver nanoparticles  is theoretically investigated and its use as a sensor is evaluated. The coherent scattering model predicts a possible collective plasmonic mode, excited in internal incidence, which can be tuned by choosing the radius $a$ of the NPs on the monolayer and its cover fraccion $\Theta$. The obtained resultsn show that a monolayer of gold NPs with radius $ a = 30 $ nm and $ \Theta = 0.125 $  can be used for biosensing. The comparison of the possible collective plasmonic mode against the plasmonic surface lattice resonances and the localized surface plasmon resonances shows comparable sensitivities between the three resonances for angles of incidence close to the critical angle.}



\end{abstracts}
%\end{abstractlongs}


% ----------------------------------------------------------------------