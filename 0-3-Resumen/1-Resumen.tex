% !TeX root = ../tesis.tex

% Thesis Abstract -----------------------------------------------------

%\begin{abstractslong}    %uncommenting this line, gives a different abstract heading
\begin{abstracts}        %this creates the heading for the abstract page
\addcontentsline{toc}{chapter}{\protect\numberline{}Resumen}

Los biosensores plasmónicos comerciales tradicionalmente emplean plasmones-polaritones de superficie excitados en una película continua de oro. Recientemente, se han propuesto arreglos nanoestructurados tanto periódicos como desordenados para el biosensado empleando resonancias plasmónicas de red de superficie  y  resonancias de plasmón de superficie localizadas,  respectivamente. Sin embargo, su fabricación es lenta y costosa, pues involucran técnicas de fabricación como la litografía coloidal de hueco-máscara, y deposición por evaporación catódica y  formación de poros. En este trabajo de tesis de licenciatura se estudia la posibilidad de emplear sistemas monocapa desordenadas de nanopartículas esféricas como biosensor. Con el modelo de esparcimiento coherente se analiza de forma teórica la reflectancia de una monocapa desordenada de nanopartículas esféricas de oro y de plata, y se evalúa su uso como sensor. El modelo de esparcimiento coherente predice un modo similar al modo plasmónico colectivo, excitado en incidencia interna, que puede sintonizarse al elegir el radio $a$ de las nanopartículas de la monocapa y su fracción de cubierta $\Theta$. Los resultados obtenidos muestran que una monocapa desordenada de nanopartículas de oro con radio $a=30$ nm y $\Theta=0.125$ podría emplearse para el biosensado al igual que una de nanopartículas de plata de $a=40$ nm y $\Theta=0.1$. La comparación del supuesto modo plasmónico colectivo predicho por el modelo de esparcimiento coherente con la resonancia plasmónica de red de superficie y la resonancia de plasmón de superficie localizada muestra sensibilidades comparables para ángulos de incidencia cercanos al ángulo crítico. Adicionalmente, en este trabajo de tesis se compara la sensibilidad del supuesto modo plasmónico colectivo con la del plasmón-polaritón de superficie,  mostrando que existen condiciones particulares en las que la sensibilidad de una monocapa de nanopartículas de oro (plata) es comparable con la de una película delgada de oro (plata).

\vspace*{1cm}\textit{
Commercial plasmonic biosensors use traditionally surface plasmon-polaritons excited on a continuous gold film. Periodic and disordered nanostructured arrays have been recently proposed for biosensing by employing plasmonic surface lattice resonances and localized surface plasmon resonances, however their production is slow and costly due their fabrication techniques such as hole-mask colloidal litography or sputter depostion followed by pore formation. In this bachelor thesis it is studied the possibility to use disordered arrays of spherical nanoparticles  as a biosensor. By using the coherent scattering model the reflectance of a  monolayer of spherical gold and silver nanoparticles  is theoretically investigated and its use as a sensor is evaluated. The coherent scattering model predicts a mode similar to a  collective plasmonic mode, excited in internal incidence, which can be tuned by choosing the radius $a$ of the nanoparticles on the monolayer and its cover fraccion $\Theta$. The obtained results show that a monolayer of gold nanoparticles with radius $ a = 30 $ nm and $ \Theta = 0.125 $  could be used for biosensing, as well as a monolayer of $a=40$ nm silver nanoparticles monolayer with $\Theta=0.1$ . The comparison of the possible collective plasmonic mode against the plasmonic surface lattice resonances and the localized surface plasmon resonances shows comparable sensitivities  for angles of incidence close to the critical angle. In addition to this thesis work, the sensibility of the possible collective plasmonic mode and the surface plasmon-polariton were compared, showing that under particular conditions the sensitivity of a gold (solver) nanoparticles monolayer is comparable to that of a thin gold (silver) film. }



\end{abstracts}
%\end{abstractlongs}


% ----------------------------------------------------------------------