\chapter*{Conclusiones}
\addcontentsline{toc}{chapter}{\protect\numberline{}Conclusiones}
\label{chapter:Conclusiones}

El objetivo principal de este trabajo fue estudiar la respuesta electromagnética (EM) de una monocapa desordenada de nanopartículas (NPs) esféricas para evaluar su uso en el biosensado. Se calculó la  reflectancia y transmitancia de la monocapa cuando una onda plana monocromática incide sobre ésta, empleando el modelo de esparcimiento coherente (Coherent Scattering Model, CSM) \cite{reyes2018analytical,pena-gomar2006coherent,barrera1991optical,garcia2012multiple} ---el cual proporciona expresiones analíticas de la reflectancia y transmitancia cuando la monocapa se encuentra embebida en un dieléctrico, denominado matriz, y soportada por un sustrato--- considerando para  la función dieléctrica de las NPs que forman la monocapa tanto el modelo de Drude-Sommerfeld como  la corrección por tamaño de la función dieléctrica del oro y de la plata para NPs esféricas. Asimismo, se comparó la sensibilidad de la respuesta EM de una monocapa de NPs esféricas de oro con la de sensores comerciales, basados en el uso del plasmón-polaritón de superfice (Surface Plasmon Polariton, SPP)  excitados en una película continua de oro de $50$ nm de grosor, y la de propuestas de biosensores nanoestructurados reportados en la literatura \cite{svedendahl2009refractometric,kabashin2009plasmonic,danilov2018ultra} basados en la resonancia de plasmón de superficie localizada (Localized Surface Plasmon Resonance, LSPR) y en la resonancia de red de superfice plasmónica (Plasmonic Surface Lattice Resonance, PSLR). Finalmente, se calculó la figura de mérito (Figure of Merit, FoM) de bulto $\textit{FoM}_B$ para el supuesto modo colectivo y su longitud de penetración.

Con base en los resultados obtenidos, se concluye que el  CSM predice la existencia de un supuesto modo plasmónico colectivo que se excita cuando la monocapa se ilumina en un esquema de reflexión total atenuada. El supuesto modo plasmónico colectivo se excita a energías menores a las de las resonancias de plasmón de superficie de partícula individual (Single Particle Surface Plasmon Resonances, SP-SPRs) de las NPs que conforman la monocapa y se observa tanto en polarización \emph{p} como \emph{s}, siendo su respuesta más intensa para polarización \emph{p}. El supuesto modo plasmónico colectivo puede sintonizarse al modificarse el radio $a$ de las NPs de la monocapa y su fracción de llenado $\Theta$: para un valor de $a$ fijo, al aumentar $\Theta$ el supuesto modo colectivo presenta una respuesta más evidente para ángulos de incidencia rasantes además de un mayor corrimiento al rojo respecto a la SP-SPR dipolar, mientas que para un valor de $\Theta$ fijo, al aumentar el radio de las NPs sólo se observa el corrimiento al rojo; para ambos casos la resonancia se ensancha. Asimismo, se observó que el supuesto modo plasmónico colectivo presenta características de un modo guiado.

Se determinó que una monocapa de NPs de oro con $a=30$ nm y $\Theta=0.125$ y una de NPs de plata con $a=40$ nm y $\Theta=0.1$ pueden emplearse como biosensor pues estos parámetros garantizan la validez del CSM y sintonizan al supuesto modo plasmónico colectivo dentro del espectro visible, además de que minimizan la reflectancia a las longitudes de onda del supuesto modo plasmónico colectivo a ángulos de incidencia $\theta_i$ menores a $80^\circ$, lo que facilitaría su identificación de forma experimental. Al variar el índice de refracción de la matriz $n_m$, el supuesto modo plasmónico colectivo se corre al rojo y al azul dependiendo del ángulo de incidencia; cuando $\theta_i\gtrapprox\theta_c$, con $\theta_c$ el ángulo crítico entre el sustrato y la matriz, se observa el corrimiento al azul, en contraste con el SPP, la LSPR y la PSLR que únicamente presentan corrimiento al rojo. A pesar de que la longitud de penetración del supuesto modo plasmónico colectivo y del SPP son del mismo orden de magnitud, el SPP es más sensible a cambios $n_m$ que el supuesto modo plasmónico colectivo sin embargo, dependiendo del intervalo de $n_m$, la sensiblidad entre los dos modos puede ser un orden de magnitud mayor (cuando se observa un corrimiento al rojo del supuesto modo colectivo) o del mismo (cuando se observa un corrimiento al azul). La sensibilidad del supuesto modo  plasmónico colectivo es semejante a la de la LSPR y a la de la PSPR, es decir, un arreglo de NPs esféricas desordenadas compite con la sensibilidad de un arreglo nanoestructurado de nanocilindros ordenados. De igual forma, se concluyó que la $\textit{FoM}_B$ para el supuesto modo plasmónico colectivo es consistente con la reportada en la literatura para arreglos nanoestructurados.

El SPP cuenta con mejores $\textit{FoM}_B$ que los arreglos nanoestructurados sin embargo, estos últimos se han sido empleados en la detección de pocas partículas al rededor de la nanoestructura empleada que cambien el índice de refracción de la matriz localmente alrededor de las NPs y no el índice de refacción de toda la matriz, por lo se ha propuesto el uso de la FoM de superficie $\textit{FoM}_S$ para cuantificar la sensibilidad del arreglo empleado al aumentar el grosor de una capa uniforme que rodea la nanoestrucura \cite{estevez2014trends,svedendahl2009refractometric}. Por tanto, se propone calcular la FoM de superficie considerando un analito particular para funcionalizar a las NPs de la monocapa y estudiar el corrimiento de la longitud de onda de excitación del supuesto modo plasmónico cuando se forma una capa uniforme al rededor de las NPs como continuación al trabajo presentado en esta tesis. De igual forma, se propone la identificación experimental del supuesto modo guiado con monocpas desordenadas de NPs esféricas de oro y de plata con los valores de $a$ y $\Theta$ propuestos para cada material. 